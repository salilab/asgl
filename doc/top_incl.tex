\TOP\ is an interpreter of a scripting language specialized for certain
areas. Its use includes programs \MODELLER\ and \ASGL. Its syntax resembles 
that of \FORTRAN.


\section{The source file}

\label{SECTION:topsource}

Each \TOP\ program or include file is stored in a file named \Z{program.top}. 
The \V{.top} extension is mandatory.

The \TOP\ program consists of a series of commands. The order of
commands is important. An example of the \TOP\ program that
writes integers 1 to 10 to the output file is:

\begin{verbatim}
# Define a variable:
DEFINE_INTEGER VARIABLES = IVAR

# Open a file for appending
OPEN IO_UNIT = 21, OBJECTS_FILE = 'output.file', FILE_ACCESS = 'APPEND'

# Loop from 1 to 10:
DO  IVAR = 1, 10, 1
  # Append IVAR to the output file:
  WRITE IO_UNIT = 21, OBJECTS = IVAR
END_DO

# Close a file
CLOSE IO_UNIT = 11

# Exit:
STOP
\end{verbatim}


There can be at most one command per line. Each command or line can be
at most \V{LENACT} (512) characters long.  The command can extend
over several lines if a continuation character `;' is used to indicate 
the end of the current line.  Everything on that line after the 
continuation character is ignored.

A comment character `\#' can be used anywhere on the line to ignore
everything that occurs after the comment character.

Blank lines are allowed. They are ignored. 

TAB characters are replaced by blank characters.

\TOP\ converts all commands to upper case, except for the string 
constants that are quoted in single quotes '. Thus, \TOP\ is case
insensitive, except for the quoted strings.

There are two groups of commands: flow control commands and commands that 
perform certain tasks. The next two sections describe the flow control 
commands and those `performing' commands that are an integral part of \TOP.  
There are also additional commands specific to each application of \TOP, 
such as \MODELLER\ and \ASGL, which are described elsewhere. 

The usual \UNIX\ conventions are used for typesetting the rules.
Table~\ref{tab:vartypes} explains the shorthand used to describe
different variables and constants:

% \vspace{5mm}

\begin{table}[hbt]
\begin{center}
\begin{tabular}{ll}
\Integer{1}\         & an integer variable or constant \\
\Real{1}\            & a real variable or constant \\
\String{1}\          & a string variable or constant \\
\Logical{1}\         & a logical variable or constant \\
\vartype{var\_}{1}   & prefix for a variable \\
\vartype{const\_}{1} & prefix for a constant \\
\Variable{1}\        & \vartype{var\_integer}{1} \OR\ \vartype{var\_real}{1} 
                       \OR\ \vartype{var\_string}{1} \OR\ 
                            \vartype{var\_logical}{1} \\
\Constant{1}\        & \vartype{const\_integer}{1} \OR\ 
                       \vartype{const\_real}{1} \OR\ 
                       \vartype{const\_string}{1} \OR\ 
                       \vartype{const\_logical}{1} \\
\Number{1}\          & \Integer{1}  \OR\ \Real{1} \\
\Quantity{1}\        & \Variable{1} \OR\ \Constant{1} \\
\Quantity{0}\        & a vector of any length with elements \Quantity{1} \\
\Quantity{N}\        & a vector of $N$ elements \Quantity{1}\\
\end{tabular}
\end{center}
\newcommand{\vartypes}{{\em List of variable types in \TOP.}}
\caption[\vartypes]
{\latex{\vartypes}}
\label{tab:vartypes}
\underfigtab
\end{table}

All the variables are formally vectors. When a variable is referred to
in a scalar context its first element is used. All elements of one
vector are of the same type. All variables, including a vector of 
the variable length, must have at least one element.

There are four different variable types: integer, real, string and
logical.

The real constant is (\FORTRAN\ real number representation):
\begin{verbatim}
  [+|-][digits][.][digits][{e|E|d|D}{+|-}digits]
\end{verbatim}

The integer constant is (\FORTRAN\ integer number representation): 
\begin{verbatim}
  [+|-][digits] 
\end{verbatim}

The logical constant can be either \V{on} or \V{off} (case insensitive).

The string constant can contain any character except for a prime
'. It can be optionally
enclosed in primes.  If it is not quoted it is converted to
upper case and its extent is determined by the position of the blanks
on each side of the contiguous string of non-blank characters.


\section{\TOP\ Commands}

There are `flow control' and `performing' commands. If general, 
the `performing' commands have the following syntax:

{\bf ACTION} [{\bf ASSIGNMENT}, {\bf ASSIGNMENT}, \ldots, {\bf ASSIGNMENT}]

{\bf ACTION} specifies what action to take. {\bf ASSIGNMENT} sets the
variable to the specified value.  The values assigned in this way are
kept until the next assignment. For example, 
\V{CALL ROUTINE = 'routine\_name', IVAR = 3} sets the integer variable 
{\bf IVAR} to 3 and then calls routine {\tt routine\_name}; if {\bf IVAR} is
not changed in the routine, its value will remain to be 3 after the call
to the routine.

There can be any number of assignments in a command. They must be
separated by commas. The assignment is of the form:

\Variable{0}\ = [-]\Quantity{0}

The `=' character is optional (can be replaced with a blank). 

\Integer{1}\ and \Real{1}\ can be assigned to each other. When a real number 
is assigned to an integer variable, the decimal places are ignored.
That is, the result is the same as if the \FORTRAN\ function \V{IFIX()} was
used. There must be no space between the optional $-$ and \Quantity{0}.
If a vector variable is assigned to a variable, all its elements
are used.

Real, integer, and logical variables can also be assigned to a string variable.
The conversion of a real variable to a string value is guided by the
\TOP\ variable \K{NUMBER\_PLACES} which is of type \Integer{2}. The 
first element of \K{NUMBER\_PLACES}
sets the number of places before the decimal point, and the second element
the number of places after the decimal point. If the latter is $-1$, an
integer number without a decimal point is obtained, if 0 there is a
decimal point without any decimal places.

Assignments can follow any command, except \C{DO}, \C{END\_DO}, 
\C{GO\_TO}, \C{LABEL}, \C{STOP}, and \C{END\_\-SUBROUTINE}.



\Command      {DEFINE\_INTEGER}{define integer variables}

\Options {
\OptLine{VARIABLES}{TYPE}{VALUES}{DEFAULT}{DESCRIPTION} 
}

\Description  {This command defines user integer variables. All 
variables used in the \TOP\ program must be defined.
An exception are the pre-defined \TOP\ variables listed 
at the end of this section.}


\Command      {DEFINE\_LOGICAL}{define logical variables}

\Options {
\OptLine{VARIABLES}{TYPE}{VALUES}{DEFAULT}{DESCRIPTION}  
}

\Description  {This command defines user logical variables.}



\Command      {DEFINE\_REAL}{define real variables}

\Options{
\OptLine{VARIABLES}{TYPE}{VALUES}{DEFAULT}{DESCRIPTION}  
}

\Description  {This command defines user real variables.}



\Command      {DEFINE\_STRING}{define string variables}

\Options {
\OptLine{VARIABLES}{TYPE}{VALUES}{DEFAULT}{DESCRIPTION} 
}

\Description  {This command defines user string variables.}




\Command      {SET}{set variable}

\Commandline {SET [ASSIGNMENT, [ASSIGNMENT, \ldots [ASSIGNMENT]]]}

\Description  {This command sets the values of variables of any of 
the four types. See the description of {\bf ASSIGNMENT} above.

There can be \UNIX\ shell environment variables in any input or output 
filename. The environment variables have to be in the format {\tt
\$\{VARNAME\}} or {\tt \$(VARNAME)}. Also, four predefined macros are 
available for string variables: 

Four predefined macros are available for string variables:

\begin{itemize}
\item \Z{\$\{LIB\}} is expanded into \V{\$LIB\_APPLICATION} shell 
      environment variable,
      where APPLICATION is the name-version of the program (\eg, MODELLER5);
\item \Z{\$\{DIR\}} is expanded into the \TOP\ variable \K{DIRECTORY},
\item \Z{\$\{JOB\}} is expanded into the root of the \TOP\ script filename,
\item \Z{\$\{DEFAULT\}} is expanded into \V{(ROOT\_NAME)(FILE\_ID)(ID1)(ID2)(FILE\_EXT)},
      where \K{ROOT\_\-NAME}, \K{FILE\_ID}, \K{ID1}, \K{ID2}, and \K{FILE\_EXT} are
      \TOP\ variables. \K{FILE\_ID} is a string that may be set to
      \Z{default}. In that case, a hard-wired short string is used instead of
      \K{FILE\_ID}. Otherwise, the explicitly specified \K{FILE\_ID} is
      applied instead.  In any case, \K{FILE\_ID} is not modified by the
      filename generation routine so that it can be used more than once
      without resetting it to the \Z{default} value. Four digits are used
      for both \K{ID1} and \K{ID2}. For example, \Z{2ptn.B99990001}
      results from \K{ROOT\_NAME} = \Z{2ptn}, \K{FILE\_EXT} = \Z{.B},
      \K{ID1} = 9999, and \K{ID2} = 1.
\end{itemize}
}

\Command      {OPERATE}{perform mathematic operation}

\Options{
\OptLine{OPERATION}{TYPE}{VALUES}{DEFAULT}{DESCRIPTION} 
\OptLine{RESULT}{TYPE}{VALUES}{DEFAULT}{DESCRIPTION} 
\OptLine{ARGUMENTS}{TYPE}{VALUES}{DEFAULT}{DESCRIPTION} 
}

\Description  {This command performs a specified mathematical operation.
There can be up to \V{MRPRM} (120) operands for the \Z{SUM} 
and \Z{MULTIPLY} operations, but only two for \Z{DIVIDE} 
and \Z{POWER}. The \K{RESULT} value has to be the name of a real variable.}



\Command      {STRING\_OPERATE}{perform string operation}

\Options{
\OptLine{OPERATION}{TYPE}{VALUES}{DEFAULT}{operation to perform: \V{CONCATENATE}} 
\OptLine{RESULT}{TYPE}{VALUES}{DEFAULT}{DESCRIPTION} 
\OptLine{STRING\_ARGUMENTS}{TYPE}{VALUES}{DEFAULT}{DESCRIPTION} 
}

\Description  {This command performs a specified string operation.
There can be up to \V{MSPRM} (130) operands for the 
\V{CONCATENATE} operation.  The \K{RESULT} value has to 
be a name of the string variable.}



\Command      {RESET}{reset \TOP}

\Description  {This command resets the internal state of
\TOP\ and its predefined variables to their initial 
values. It does this by calling the initialization 
routine that reads the \Z{top.ini} file. This command
also undefines all user defined variables.}



\Command      {OPEN}{open input file}

\Options{
\OptLine{IO\_UNIT}{TYPE}{VALUES}{DEFAULT}{DESCRIPTION} 
\OptLine{OBJECTS\_FILE}{TYPE}{VALUES}{DEFAULT}{DESCRIPTION} 
\OptLine{FILE\_ACCESS}{TYPE}{VALUES}{DEFAULT}{DESCRIPTION} 
\OptLine{FILE\_STATUS}{TYPE}{VALUES}{DEFAULT}{DESCRIPTION} 
}

\Description  {This command opens a specified file on the specified I/O 
stream for formatted access. \FORTRAN\ conventions apply
to \K{FILE\_ACCESS} and \K{FILE\_STATUS}.}




\Command      {WRITE}{write \TOP\ objects}

\Options{
\OptLine{IO\_UNIT}{TYPE}{VALUES}{DEFAULT}{DESCRIPTION} 
\OptLine{OBJECTS}{TYPE}{VALUES}{DEFAULT}{DESCRIPTION} 
\OptLine{NUMBER\_PLACES}{TYPE}{VALUES}{DEFAULT}{DESCRIPTION} 
\OptLine{OUTPUT\_DIRECTORY}{TYPE}{VALUES}{DEFAULT}{DESCRIPTION} 
}

\Description  {This command writes the specified objects to a single line 
which is then written to a selected I/O stream. 
Each element of the \K{OBJECTS} vector is first tested
if it is a name of a variable of any type. If it is
the contents of that variable is written out. If it is 
not, the element is treated as a string constant.
The first and second element of \K{NUMBER\_PLACES} set the 
numbers of places before and after the decimal point,
respectively, for real and integer objects.}





\Command      {READ}{read record from input file}

\Options {
\OptLine{IO\_UNIT}{TYPE}{VALUES}{DEFAULT}{DESCRIPTION} 
\OptLine{RECORD}{TYPE}{VALUES}{DEFAULT}{DESCRIPTION} 
}

\Description  {This command reads a line from the file on the I/O channel 
\K{IO\_UNIT}. The line goes into the string variable \K{RECORD}.}


\Command      {CLOSE}{close an input file}

\Options {
\OptLine{IO\_UNIT}{TYPE}{VALUES}{DEFAULT}{DESCRIPTION} 
}

\Description  {This command closes a specified I/O stream.}




\Command      {WRITE\_TOP}{write the \TOP\ program}

\Options{
\OptLine{FILE}{TYPE}{VALUES}{DEFAULT}{DESCRIPTION} 
\OptLine{OUTPUT\_DIRECTORY}{TYPE}{VALUES}{DEFAULT}{DESCRIPTION} 
\OptLine{FILE\_ACCESS}{TYPE}{VALUES}{DEFAULT}{DESCRIPTION} 
}

\Description  {This command writes the current \TOP\ program in memory to a
               specified file.}




\Command      {SYSTEM}{execute system command}

\Options {
\OptLine{COMMAND}{TYPE}{VALUES}{DEFAULT}{DESCRIPTION} 
}

\Description  {This command executes the specified UNIX command.}


\Command      {INQUIRE}{check if file exists}

\Options {
\OptLine{FILE}{TYPE}{VALUES}{DEFAULT}{DESCRIPTION} 
}

\Description  {This command assigns 1 to \K{FILE\_EXISTS} if the 
specified file exists,
otherwise it assigns 0. You can use it with a subsequent 
\C{IF} command for the flow control.}



\Command      {GO\_TO}{jump to label}

\Commandline  {GO\_TO \String{1}}

\Description  {The `go\_to' statement, which transfers execution to the 
\TOP\ statement occurring after the \C{LABEL} statement with
the same name.}




\Command      {LABEL}{place jump label}

\Commandline  {LABEL \String{1}}

\Description  {This command labels a target position for the \C{GO\_TO} 
statement with the same name.}



\Command      {INCLUDE}{include \TOP\ file}

\Options {
\OptLine{INCLUDE\_FILE}{TYPE}{VALUES}{DEFAULT}{DESCRIPTION} 
}

\Description  {This command includes a \TOP\ file \K{INCLUDE\_FILE}. 
You do not have to specify the {\tt .top} extension. 
First, the given filename is
tried. Second, the directory specified in the {\tt 
\$BIN\_APPLICATION} environment variable is prefixed and 
the open function is tried again. \C{INCLUDE} command is 
useful for including standard subroutines.}



\Command      {CALL}{call \TOP\ subroutine}

\Options {
\OptLine{ROUTINE}{TYPE}{VALUES}{DEFAULT}{DESCRIPTION} 
}

\Description  {This command calls a \TOP\ subroutine \K{ROUTINE}.}




\Command      {SUBROUTINE}{define \TOP\ subroutine}

\Options {
\OptLine{ROUTINE}{TYPE}{VALUES}{DEFAULT}{DESCRIPTION} 
}

\Description  {This command is the first \TOP\ statement for any routine. 
It has to have a matching \C{END\_\-SUBROUTINE}. No nesting of subroutine
definitions is allowed, although the definitions can be located anywhere 
in a file.}



\Command      {RETURN}{return from \TOP\ subroutine}

\Description  {This command will exit the execution from the current routine. 
It is optional.}




\Command      {END\_SUBROUTINE}{end definition of \TOP\ subroutine}

\Description  {This command has to be present at the end of each routine. 
Possibly used instead of \C{RETURN}
if \C{RETURN} not present.}



\Command      {DO}{DO loop}

\Commandline  
{
\C{DO} {\bf VAR} = \V{START}, \V{END}, \V{STEP} \\
  \hspace{2.0cm} commands \\
\C{END\_DO}
}

\Description{Commas after \V{START} and \V{END} can be omitted. This loop 
is exactly like a \FORTRAN\ \V{DO} loop except that real
values are allowed for any of the four controlling 
variables. {\bf VAR} must be a variable, while \V{START}, 
\V{END} and \V{STEP} can also be constants.}


\Command      {IF}{conditional statement for numbers}

\Options{
\OptLine{OPERATION}{TYPE}{VALUES}{DEFAULT}{\V{EQ} \OR\ \V{GT} \OR\ \V{LT} \OR\
         \V{GE} \OR\ \V{LE} \OR\ \V{NE}} \\
}

\Description  {This command performs conditional IF operation on two
real arguments. The possible operations are equal
(\V{EQ}), greater than (\V{GT}), less than (\V{LT}), greater or
equal (\V{GE}), less or equal (\V{LE}), and not equal
(\V{NE}). If the condition
is true, the command specified in the \K{THEN} variable
is executed. Otherwise the command in the \K{ELSE}
variable is executed. Typically, these commands are 
\C{GO\_TO} statements.}



\Command      {STRING\_IF}{conditional statement for strings}

\Options{
\OptLine{OPERATION}{TYPE}{VALUES}{DEFAULT}{\V{EQ} \OR\ \V{NE}\ \OR\ \V{INDEX}} 
\OptLine{STRING\_ARGUMENTS}{TYPE}{VALUES}{DEFAULT}{DESCRIPTION} 
\OptLine{THEN}{TYPE}{VALUES}{DEFAULT}{DESCRIPTION} 
\OptLine{ELSE}{TYPE}{VALUES}{DEFAULT}{DESCRIPTION} 
}

\Description  {This command performs conditional IF operation on two
string arguments. The possible operations are equal
(\V{EQ}), not equal (\V{NE}), and the \FORTRAN\ {\tt index()}
function (\V{INDEX}), which returns true if there is 
`argument2' substring within `argument1'. If the condition
is true, the command specified in the \K{THEN} variable
is executed. Otherwise the command in the \K{ELSE}
variable is executed. Typically, these commands are 
\C{GO\_TO} statements.}


\Command      {STOP}{exit \TOP}

\Description  {This command stops the execution of the \TOP\ program.}


{
\section{Predefined \TOP\ variables}
\samepage
\vspace{5mm}
\begin{table}[hbt]
\begin{center}

\begin{tabular}{ll} \hline
Name              &  Type         \\ \hline
\K{ARGUMENTS}         &  \Real{0}     \\
\K{IO\_UNIT}          &  \Integer{1}  \\
\K{ID1}               &  \Integer{1}  \\
\K{ID2}               &  \Integer{1}  \\
\K{NUMBER\_PLACES}    &  \Integer{2}  \\
\K{FILE\_EXISTS}      &  \Integer{1}  \\
\K{OUTPUT\_CONTROL}   &  \Integer{4}  \\
\K{STOP\_ON\_ERROR}   &  \Integer{1}  \\
\K{ERROR\_STATUS}     &  \Integer{1}  \\
\K{OBJECTS}           &  \String{0}   \\
\K{VARIABLES}         &  \String{0}   \\
\K{ROUTINE}           &  \String{1}   \\
\K{ROOT\_NAME}        &  \String{1}   \\
\K{DIRECTORY}         &  \String{1}   \\
\K{FILE\_ID}          &  \String{1}   \\
\K{OPERATION}         &  \String{1}   \\
\K{RESULT}            &  \String{1}   \\
\K{STRING\_ARGUMENTS} &  \String{0}   \\
\K{OBJECTS\_FILE}     &  \String{1}   \\
\K{INCLUDE\_FILE}     &  \String{1}   \\
\K{FILE}              &  \String{1}   \\
\K{RECORD}            &  \String{1}   \\
\K{THEN}              &  \String{1}   \\
\K{ELSE}              &  \String{1}   \\
\K{COMMAND}           &  \String{1}   \\
\K{FILE\_EXT}         &  \String{1}   \\
\K{OUTPUT\_DIRECTORY} &  \String{1}   \\
\K{FILE\_ACCESS}      &  \String{1}   \\
\K{FILE\_STATUS}      &  \String{1}   \\ \hline
\end{tabular}
\end{center}
\newcommand{\predefinedvar}{{\em Predefined \TOP\ variables}}
\caption[\predefinedvar]
{\latex{\predefinedvar}}
\label{tab:predefinedvar}
\underfigtab
\end{table}
}
