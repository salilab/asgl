\section{\ASGL\ commands}

\Command      {\#EPSF}{produce encapsulated PostScript}
\Commandline  {\#EPSF [ .ps \OR\ .eps ]}
\Description  {
If the very first line in the steering file starts with the above in the
first column, \ASGL\ will
create a PS file that conforms to the Encapsulated PostScript standard. 
In this case, only one page graphics per file can be produced. Optionally, 
you can also specify the extension for the PS filename. It is usually 
{\tt .eps} for the Encapsulated PostScript, but you may want to keep it 
as {\tt .ps} if you depend on it in some other programs. The default is 
{\tt .ps}. You can only specify the extension when \C{\#EPSF} is the first 
line. If you produce an EPSF file you can include it properly in the 
\LaTeX\ document using: \\
% \begin{center}
% {\tt 
% \\begin\{center\} \\
%   \\leavevmode \\
%   \\def\epsfsize\#1\#2\{scaling\_factor \#1\} \\
%   \\epsffile\{file\_name.ps\} \\
% \\end\{center\}
% \end{center}
% }
}


\Command      {SET STAMP\_TEXT}{stamp the page}
\Commandline  {SET STAMP\_TEXT = \String{1} \hspace{1cm}
                                 [ \V{DEFAULT} \OR\ \V{NONE} 
                                   \V{your text}], STAMP\_SIZE = \Real{1} }
\Description  {If set to \V{DEFAULT} the page will be stamped at the bottom 
right
corner with the \ASGL\ version, date, time, filename, and page number.
The default value is \V{DEFAULT} for PostScript and \V{NONE} for Encapsulated
PostScript. The command is not allowed in Encapsulated PostScript.
}


\Command      {READ\_TABLE}{read the Table array of data}

\Options{
\OptLine{FILE}{TYPE}{VALUES}{DEFAULT}{DESCRIPTION} \\ 
\OptLine{ADD\_DATA}{TYPE}{VALUES}{DEFAULT}{DESCRIPTION} \\
\OptLine{POINT\_MODULUS}{TYPE}{VALUES}{DEFAULT}{DESCRIPTION} \\
\OptLine{ROW\_RANGE}{TYPE}{VALUES}{DEFAULT}{DESCRIPTION} \\
}
\Description{
This command will read a text file containing a rectangular array
of values. The columns represent vectors that can be later
used for plotting as X and Y coordinates. This array will be referred
to as the Table array. If \K{ADD\_DATA[1]} = \V{ON} new columns will be appended to
the right of the existing columns. The new number of data points is the
number of rows in the new file. If \K{ADD\_DATA[2]} = \V{ON} new rows will be 
appended to the end of the existing rows. The new number of columns is the
number of columns in the new file. The command will only read the points
whose index satisfies $\mbox{mod}(i-1, \K{POINT\_MODULUS}) = 0$ and is
within boundaries of \K{ROW\_RANGE}. \K{ROW\_RANGE} can be 0 or -999 to
indicate that it is to be ignored. The 
total number of points read is approximately for a factor of
\K{POINT\_MODULUS} smaller than the total number of points between selected
rows in a file. This option is useful for plotting very large files.
The line can start with `\#', in which case it will be ignored as far
as the points are concerned. There can be string columns in the file
that can contain data for point labelling by \C{PLOT2D}.
Number/string column identification is achieved in one of the two 
ways: If there is a comment line `\#COLUMNS' before any non-comment line, 
then that line is used for identification. 
The comment line must contain a series of `N' for number columns 
and `S' for string columns, in the correct order. 
Otherwise, the items from the 
first non-comment line are interpreted as numbers or strings. 
A string is everything that cannot be interpreted as a number. If you
want blanks in strings, quote them in single quotes '. The last line
starting with `\#CAPTION\_TEXT' is assigned to \K{CAPTION\_TEXT}, which is
useful for automated production of titles. Similarly for
`\#PRINT\_TEXT' and `\#DESCRIPTION'.
}


\Command      {WRITE\_TABLE}{write the Table array of data}

\Options{
\OptLine{FILE}{TYPE}{VALUES}{DEFAULT}{DESCRIPTION} \\
\OptLine{NO\_XY\_SCOLUMNS}{TYPE}{VALUES}{DEFAULT}{DESCRIPTION} \\
\OptLine{ XY\_SCOLUMNS}{TYPE}{VALUES}{DEFAULT}{DESCRIPTION} \\
}
\Description  {
It will write a text file containing an array of values, where columns
are selected from the Table array using the \K{NO\_XY\_SCOLUMNS} and
\K{XY\_SCOLUMNS} variables.  See \C{WORLD} for explanation of
these variables.}



\Command      {READ\_DPLOT}{read the Density array of data}

\Options{
\OptLine{FILE}{TYPE}{VALUES}{DEFAULT}{DESCRIPTION} \\
\OptLine{DPLOT\_FORMAT}{TYPE}{VALUES}{DEFAULT}{DESCRIPTION} \\
\OptLine{DPLOT\_COLUMN}{TYPE}{VALUES}{DEFAULT}{DESCRIPTION} \\
\OptLine{DPLOT\_SYMMETRIZE}{TYPE}{VALUES}{DEFAULT}{DESCRIPTION} \\
\OptLine{DPLOT\_FILL}{TYPE}{VALUES}{DEFAULT}{DESCRIPTION} \\
\OptLine{DPLOT\_ORIENTATION}{TYPE}{VALUES}{DEFAULT}{DESCRIPTION} \\
% \K{DPLOT\_ORIENTATION}  = \String{1}  & XY \OR\ YX\\
}
\Description{It will read a text file containing the Density
array of data.  Two formats are possible. In free format
(\K{DPLOT\_FORMAT}=\V{OFF}), a text file must contain two integer dimensions
(NX,NY) followed by data NX * NY values. The first line can contain NX and
NY, at most. The following lines can contain a single row of an array,
at most. The order of reading the elements is:

{\tt read(ioinp,*)NX,NY \\
   do  i = 1, NX \\
     read(ioinp,*) (array(i,j),j=1,NY) \\
   end do
}

This array can then be plotted using \C{DPLOT} command. The 
array(1,1) element will appear at the coordinate system origin, in the
lower left corner of the coordinate system. When
formatted input is used (\K{DPLOT\_FORMAT}=\V{ON}), a text file contains any
number of lines in the form:

   I, J, \V{number\_1} \V{number\_2}, ...,  \V{number\_DPLOT\_COLUMN}, ...

These lines are read and the Density array is filled in on the fly as 
specified by the I and J indices and the value in column \K{DPLOT\_COLUMN}.
All other numbers in a line are ignored. The elements of the Density 
array not assigned explicitly are set to \K{DPLOT\_FILL}. 

With the formatted input, when \K{DPLOT\_SYMMETRIZE} = \V{ON}, the array 
is also symmetrized, \ie\ array(j,i)=array(i,j) for each line that is read in.

If the \K{DPLOT\_ORIENTATION} is \V{YX}, instead of \V{XY}, the X and Y
axes are swapped during reading in the data.
}




\Command      {WRITE\_DPLOT}{write the Density array of data}

\Options{
\OptLine{FILE}{TYPE}{VALUES}{DEFAULT}{DESCRIPTION} \\
\OptLine{DPLOT\_FORMAT}{TYPE}{VALUES}{DEFAULT}{DESCRIPTION} \\
\OptLine{ DPLOT\_ORIENTATION}{TYPE}{VALUES}{DEFAULT}{DESCRIPTION} \\
% \K{DPLOT\_ORIENTATION}   = \String{1}  & XY \OR\ YX\\
}

\Description{It will write a text file containing the Density
array of data.  Two formats are possible. In free format
(\K{DPLOT\_FORMAT}=\V{OFF}), the text file will contain two integer dimensions
(NX,NY) followed by data NX * NY values. The first line will contain NX and
NY, at most. The following lines will contain a single row of an array,
at most. The order of writing the elements is:

{\tt 
   read(ioinp,*)NX,NY \\
   do  i = 1, NX \\
     read(ioinp,*) (array(i,j),j=1,NY) \\
   end do
}

When
formatted output is used (\K{DPLOT\_FORMAT}=\V{ON}), the text file will contain
a number of lines in the form:

   I, J, ARRAY

If the \K{DPLOT\_ORIENTATION} is \V{YX}, instead of \V{XY}, the X and Y
axes are swapped during writing out the data.
}



\Command      {WORLD}{define the extent of your data}

\Options{
\OptLine{PERSPECTIVE}{TYPE}{VALUES}{DEFAULT}{DESCRIPTION} \\
\OptLine{EYE\_TO\_SCR}{TYPE}{VALUES}{DEFAULT}{DESCRIPTION} \\
\OptLine{SCR\_TO\_TOP}{TYPE}{VALUES}{DEFAULT}{DESCRIPTION} \\
\OptLine{PAPER\_WINDOW}{TYPE}{VALUES}{DEFAULT}{DESCRIPTION} \\
\OptLine{POSITION}{TYPE}{VALUES}{DEFAULT}{DESCRIPTION} \\
\OptLine{WORLD\_WINDOW}{TYPE}{VALUES}{DEFAULT}{DESCRIPTION} \\
\OptLine{WORLD\_FRACTION}{TYPE}{VALUES}{DEFAULT}{DESCRIPTION} \\
\OptLine{NO\_XY\_SCOLUMNS}{TYPE}{VALUES}{DEFAULT}{DESCRIPTION} \\
\OptLine{XY\_SCOLUMNS}{TYPE}{VALUES}{DEFAULT}{DESCRIPTION} \\
\OptLine{ A4\_WINDOW\_MARGIN}{TYPE}{VALUES}{DEFAULT}{DESCRIPTION} \\
}
\Description  {
This command has to be called before any plotting is done. It
calculates the extent of the plot and initializes the Plot coordinate
system --- it determines its position on a paper. If you selected
automatic World bounds setting, you should have your data already
read by the \C{READ\_TABLE}, \C{READ\_DPLOT}, or \C{READ\_PDB} commands.

When plotting a PDB structure, the \K{PERSPECTIVE}, \K{EYE\_TO\_SCR}, and
\K{SCR\_TO\_TOP} have to be set up at the time of drawing with the
\C{BALL\_STICK} command to properly scale the plot.

   \K{PAPER\_WINDOW}
   defines the position and orientation of a window on the paper. 
   The format is (XMIN YMIN XMAX YMAX ORIENTATION). The origin for
   rotation is the origin of the Base PostScript coordinate system 
   which is in the lower left corner of a paper. Units are cm and
   degrees. This 
   window on the paper will contain the data from \K{WORLD\_WINDOW}. Any 
   data points outside this area will be clipped. If orientation of 
   the window is 90\degr\ then the plot will be printed in the 
   landscape mode.

   In \K{POSITION}, 
   if the first element is different from 0 then the \K{PAPER\_WINDOW} is 
   defined automatically using the position codes (1--37) irrespective 
   of the value of \K{PAPER\_WINDOW}. The following convention is used:

   \begin{description}
     \item[1 --  8]  small 1-4, left column; 5-8, right column (8 plots / page)
     \item[9 -- 11]  medium small (3 plots / page)
     \item[12 -- 13]  medium (2 plots / page)
     \item[14]  medium large (1 plots / page)
     \item[15]  large, landscape (1 plots / page)
     \item[16 -- 37]  tiny, portrait (32 plots / page)
   \end{description}

   If the second element is 0, the aspect ratio between X and Y is set
   to 1.3333 (horizontal rectangle), if it is 1 the aspect ratio is 1.0 
   (square), if it is 2 the aspect ratio is 3.9 (very extended horizontal 
   rectangle).

   \K{WORLD\_WINDOW}
   defines the coordinate axes in the World of the data to be 
   plotted. These ranges will correspond to the \K{PAPER\_WINDOW} window. 
   However, if any of the four values is -999 (by default this is the 
   case), the program will try to calculate that value from the data 
   read in by the last data input command (\C{READ\_TABLE},
   \C{READ\_DPLOT}, or \C{READ\_PDB}).

   \K{WORLD\_FRACTION}
   defines the fraction of the number of the central points that are
   used to get the \K{WORLD\_WINDOW} when calculated automatically.
   By default, this is 1. This is useful when there are a small number
   of outliers that you do not want to plot because the large scale
   would observe the relationship between the remaining points.

   \K{NO\_XY\_SCOLUMNS}
   defines the numbers of selected columns in the Table array that 
   are to be examined for the maximal and minimal values of X and Y, 
   respectively, when calculating the real World extent of the graph. 
   If any of the numbers is 0 then \K{NO\_XY\_SCOLUMNS} and \K{XY\_SCOLUMNS}
   are set to reflect the values in \K{XY\_COLUMNS}.
   Use this variable when automatic boundaries are required
   for multi-line plots and it is not clear which data vector should
   be used for bounds setup.

   In multi-plot plots, do not forget that undefined \K{NO\_XY\_SCOLUMNS}
   and \\
   \K{XY\_SCOLUMNS} are set automatically in the first \C{WORLD} 
   command and will stay defined until reset with \C{SET} or \C{RESET}
   commands.

   \K{XY\_SCOLUMNS}
   defines the columns for the range searching (see also above).
   The first \K{NO\_XY\_SCOLUMNS(1)} integers define the X-columns 
   in the Plotting
   array for X range and the last \K{NO\_XY\_SCOLUMNS(2)} integers are the 
   Y columns for the Y range.

   If \K{A4\_WINDOW\_MARGIN} is \V{ON} a line around the Bounding Box of the 
   Base PostScript coordinate system is drawn, \ie\ A4 paper is bounded.
}   



\Command      {AXES2D}{draw coordinate axes, ticks, and labels}

\Options{
\OptLine{Y\_SCALE}{TYPE}{VALUES}{DEFAULT}{DESCRIPTION} \\
\OptLine{CAPTION\_XLEFT}{TYPE}{VALUES}{DEFAULT}{DESCRIPTION} \\
\OptLine{CAPTION\_XRIGHT}{TYPE}{VALUES}{DEFAULT}{DESCRIPTION} \\
\OptLine{TICK\_FONT}{TYPE}{VALUES}{DEFAULT}{DESCRIPTION} \\
\OptLine{X\_LABEL\_STYLE}{TYPE}{VALUES}{DEFAULT}{DESCRIPTION} \\
\OptLine{Y\_LABEL\_STYLE}{TYPE}{VALUES}{DEFAULT}{DESCRIPTION} \\
\OptLine{X\_LABELS}{TYPE}{VALUES}{DEFAULT}{DESCRIPTION} \\
\OptLine{Y\_LABELS}{TYPE}{VALUES}{DEFAULT}{DESCRIPTION} \\
\OptLine{X\_TICK}{TYPE}{VALUES}{DEFAULT}{DESCRIPTION} \\
\OptLine{Y\_TICK}{TYPE}{VALUES}{DEFAULT}{DESCRIPTION} \\
\OptLine{X\_LABEL\_SHIFT}{TYPE}{VALUES}{DEFAULT}{DESCRIPTION} \\
\OptLine{Y\_LABEL\_SHIFT}{TYPE}{VALUES}{DEFAULT}{DESCRIPTION} \\
\OptLine{X\_TICK\_LABEL}{TYPE}{VALUES}{DEFAULT}{DESCRIPTION} \\
\OptLine{Y\_TICK\_LABEL}{TYPE}{VALUES}{DEFAULT}{DESCRIPTION} \\
\OptLine{X\_TICK\_DECIMALS}{TYPE}{VALUES}{DEFAULT}{DESCRIPTION} \\
\OptLine{Y\_TICK\_DECIMALS}{TYPE}{VALUES}{DEFAULT}{DESCRIPTION} \\
\OptLine{Y\_AXIS\_FACTOR}{TYPE}{VALUES}{DEFAULT}{DESCRIPTION} \\
\OptLine{Y\_AXIS\_FACTOR}{TYPE}{VALUES}{DEFAULT}{DESCRIPTION} \\
\OptLine{AXES2D\_LINE\_TYPE}{TYPE}{VALUES}{DEFAULT}{DESCRIPTION} \\
\OptLine{TICK\_LINE\_TYPE}{TYPE}{VALUES}{DEFAULT}{DESCRIPTION} \\
\OptLine{TICK\_SMALL\_LINE\_TYPE}{TYPE}{VALUES}{DEFAULT}{DESCRIPTION} \\
\OptLine{EXPONENT}{TYPE}{VALUES}{DEFAULT}{DESCRIPTION} \\
% \K{Y\_SCALE}          = \String{1} & \V{LEFT} \OR\ \V{RIGHT} \\
}

\Description  {
It will plot the axes of the coordinate system.

   \K{TICK\_FONT} sets the font size for labeling the ticks.

   \K{X\_LABEL\_STYLE} and \K{Y\_LABEL\_STYLE} select the labelling regime:
   \begin{description}
   \item[1] labels for X,Y-ticks supplied explicitly to the routine.
   \item[2] labels calculated automatically.
   \item[3] labels not displayed at all.
   \end{description}

   \K{X\_TICK} and \K{Y\_TICK} define, in World coordinates, the position 
   of the first tick on the X,Y-axes, the spacing between the ticks, and the 
   rightmost tick position.

   \K{X\_LABEL\_SHIFT} and \K{Y\_LABEL\_SHIFT} shift the X/Y-axes 
   labels. Shifts are specified in the Plot coordinates.

   \K{X\_TICK\_LABEL} and \K{Y\_TICK\_LABEL} set the index of the X/Y-axes 
   ticks that are numbered first, and also every which tick from the first 
   one on is numbered. 

   \K{X\_TICK\_DECIMALS} and \K{Y\_TICK\_DECIMALS} set the number of decimal 
   places used in the automatic calculation of the X/Y-axes tick
   labels. If 0, only a dot will appear after an integer. If $-1$ only
   an integer will appear.

   If \K{Y\_SCALE} is \V{RIGHT} it will plot the Y-axis ticks and their 
   numbers on the right side of the plot. Default is \V{LEFT}.

   \K{CAPTION\_XLEFT} and \K{CAPTION\_XRIGHT} are returned by \C{AXIS2D} 
   for later use by the \C{CAPTION} command in placing the captions next
   to the Y-axis. If these automatic values are not good you can correct 
   them manually (rarely needed).

   \K{X\_AXIS\_FACTOR} is used to scale the X-label ticks before the
   label is written out. If \K{EXPONENT} is \V{ON} then `En' is
   added to the tick label where 
   $n = \mbox{nint}[\log10(\mbox{\K{X\_AXIS\_FACTOR}})]$. 
   If \K{EXPONENT} is \V{OFF} then `En' is not added and could be
   included in the axis label with the \C{CAPTION CAPTION\_POSITION = 4 or 5}
   command.
}




\Command      {PLOT2D}{draw a 2D line or scatter plot}

\Options{
\OptLine{PLOT2D\_SYMBOL\_TYPE}{TYPE}{VALUES}{DEFAULT}{DESCRIPTION} \\
\OptLine{PLOT2D\_LINE\_TYPE}{TYPE}{VALUES}{DEFAULT}{DESCRIPTION} \\
\OptLine{XY\_COLUMNS}{TYPE}{VALUES}{DEFAULT}{DESCRIPTION} \\
\OptLine{POINT\_FONT}{TYPE}{VALUES}{DEFAULT}{DESCRIPTION} \\
\OptLine{LABEL\_FONT}{TYPE}{VALUES}{DEFAULT}{DESCRIPTION} \\
\OptLine{LABEL\_LOCATION}{TYPE}{VALUES}{DEFAULT}{DESCRIPTION} \\
\OptLine{LABEL\_COLUMN}{TYPE}{VALUES}{DEFAULT}{DESCRIPTION} \\
\OptLine{COLOR\_COLUMN}{TYPE}{VALUES}{DEFAULT}{DESCRIPTION} \\
\OptLine{COLOR\_RANGE}{TYPE}{VALUES}{DEFAULT}{DESCRIPTION} \\
\OptLine{COLOR\_STYLE}{TYPE}{VALUES}{DEFAULT}{DESCRIPTION} \\
}
\Description  {
This command will plot a line and/or the points defined by the selected
columns in the Table array.

   \K{PLOT2D\_SYMBOL\_TYPE} defines the symbol to be plotted for every point. 
   If 0 nothing is plotted. If set to $-1$, then the centered 
   integer indices are plotted for each point.

   \K{PLOT2D\_LINE\_TYPE} defines a line type to be plotted between successive 
   points in the Table array. If 0 no line is plotted --- used for 
   scatter plots.

   \K{XY\_COLUMNS} selects X and Y columns in the Table array. If any of 
   the two columns is not defined, it is substituted by a vector 
   $1,2,\ldots,N$.

   \K{POINT\_FONT} selects the font for the point symbol in the case where \\
   \K{PLOT2D\_SYMBOL\_TYPE} = $-1$.

   If \K{LABEL\_COLUMN} is a string column, then the labels in that
   column are drawn for each point. If \K{LABEL\_LOCATION} is 1,
   the label is centered on the point, if 2 the label is to the right
   of the point. \K{LABEL\_FONT} is the font type for the labels.

   If \K{COLOR\_COLUMN} is defined, it is used to color
   the symbols from red, yellow, green to blue in the range 
   from \K{COLOR\_RANGE}[1]
   to [2]. The HSB color convention is used (hue from 0.0 to 0.5).
   If \K{COLOR\_STYLE} is \V{GRAY}, gradual coloring is done;
   if \K{BLACK}, the values within the range are of color 0 (red),
   and those outside the range are color 0.5 (blue).
}





\Command      {SPECTRUM}{draw a bar code plot}

\Options{
\OptLine{PLOT2D\_LINE\_TYPE}{TYPE}{VALUES}{DEFAULT}{DESCRIPTION} \\
\OptLine{BAR\_XSHIFT}{TYPE}{VALUES}{DEFAULT}{DESCRIPTION} \\
\OptLine{BAR\_WIDTH}{TYPE}{VALUES}{DEFAULT}{DESCRIPTION} \\
\OptLine{XY\_COLUMNS}{TYPE}{VALUES}{DEFAULT}{DESCRIPTION} \\
}
\Description  {
This command plots an energy spectrum looking like a vertical bar code.

    \K{PLOT2D\_LINE\_TYPE} defines the line type to be used for horizontal 
    energy levels.

    \K{XY\_COLUMNS} selects the Y column in the Table array that specifies 
    energy levels. Note that Y column is plotted, not X column. This is to be 
    consistent with the \C{WORLD} command.

    \K{BAR\_XSHIFT} specifies the starting X of the energy levels in 
    the World coordinates.

    \K{BAR\_WIDTH} defines a relative width of the bars in the World 
    coordinates.
}




\Command      {HIST2D}{draw a 2D histogram}

\Options{
\OptLine{BAR\_GRAYNESS}{TYPE}{VALUES}{DEFAULT}{DESCRIPTION} \\
\OptLine{BAR\_WIDTH}{TYPE}{VALUES}{DEFAULT}{DESCRIPTION} \\
\OptLine{BAR\_LINE\_TYPE}{TYPE}{VALUES}{DEFAULT}{DESCRIPTION} \\
\OptLine{NO\_XY\_SCOLUMNS}{TYPE}{VALUES}{DEFAULT}{DESCRIPTION} \\
\OptLine{XY\_SCOLUMNS}{TYPE}{VALUES}{DEFAULT}{DESCRIPTION} \\
\OptLine{XY\_COLUMNS}{TYPE}{VALUES}{DEFAULT}{DESCRIPTION} \\
\OptLine{BAR\_XSHIFT}{TYPE}{VALUES}{DEFAULT}{DESCRIPTION} \\
}
\Description{
    This command plots a histogram of X and Y columns in the Table array.

    \K{BAR\_GRAYNESS} defines the grayness of the bar on the scale from 
    0.0 (black) to 1.0 (white).

    \K{BAR\_WIDTH} defines a relative width of the bars where 1.0 would 
    make two neighbouring bars touch each other. If less than 1 there is 
    empty space between bars.

    \K{BAR\_LINE\_TYPE} defines a linetype used to border the bar in the $\Pi$
    shaped way. If linetype is 0 bordering is not done.

    In \K{NO\_XY\_SCOLUMNS}, the first element has to be 1 because there can 
    only be one X-column.
    The second element is the number of Y-columns. It is 1 for normal
    histograms and more than 1 for a histogram where bars are stacked on
    top of each other to get a stacked bar at a single X value.

    \K{XY\_SCOLUMNS} specifies X-column and Y-columns in the Table array 
    for the histogram. The dimension of \K{XY\_SCOLUMNS} has to be 
    \K{NO\_XY\_SCOLUMNS(1)} + \K{NO\_XY\_SCOLUMNS(2)}. The default values 
    for \K{NO\_XY\_SCOLUMNS} and \K{XY\_SCOLUMNS} (when the inputs are 0) 
    are obtained from \K{XY\_COLUMNS}.  If the X-column is not defined, 
    it is substituted by a vector 
    1,2,...,N. The X coordinate of the bar specifies its mid-point 
    (not the left edge, for example). X-interval corresponding to
    one bar is always calculated automatically as the difference 
    between the first and last X divided by the number of bars less 1. 
    This works well when you have equal spacing between the points on X 
    axis. If not you can always correct the bar width using \K{BAR\_WIDTH}.

    \K{XY\_COLUMNS} is used only when default values for 
    \K{XY\_SCOLUMNS} are required.

    \K{BAR\_XSHIFT} shifts the bars for this amount along the X-axis. This is 
    to allow the plotting of several bars at the same X without
    modifying the data files.
}





\Command      {DPLOT}{draw a density plot}

\Options{
\OptLine{BAR\_LEGEND}{TYPE}{VALUES}{DEFAULT}{DESCRIPTION} \\
\OptLine{BAR\_LEGEND\_PLACES}{TYPE}{VALUES}{DEFAULT}{DESCRIPTION} \\
\OptLine{DPLOT\_GRAYNESS}{TYPE}{VALUES}{DEFAULT}{DESCRIPTION} \\
\OptLine{DPLOT\_COLOR}{TYPE}{VALUES}{DEFAULT}{DESCRIPTION} \\
\OptLine{DPLOT\_LINE\_TYPE}{TYPE}{VALUES}{DEFAULT}{DESCRIPTION} \\
\OptLine{DPLOT\_PART}{TYPE}{VALUES}{DEFAULT}{DESCRIPTION} \\
\OptLine{DPLOT\_BOUNDS}{TYPE}{VALUES}{DEFAULT}{DESCRIPTION} \\
\OptLine{DPLOT\_STYLE}{TYPE}{VALUES}{DEFAULT}{DESCRIPTION} \\
\OptLine{NUMBER\_DENSITY\_PLOT}{TYPE}{VALUES}{DEFAULT}{DESCRIPTION} \\
\OptLine{NUMBER\_PLACES}{TYPE}{VALUES}{DEFAULT}{DESCRIPTION} \\
\OptLine{POINT\_FONT}{TYPE}{VALUES}{DEFAULT}{DESCRIPTION} \\
\OptLine{PRINT\_FONT}{TYPE}{VALUES}{DEFAULT}{DESCRIPTION} \\
\OptLine{PRINT\_DXY}{TYPE}{VALUES}{DEFAULT}{DESCRIPTION} \\
\OptLine{DPLOT\_PART}{TYPE}{VALUES}{DEFAULT}{DESCRIPTION} \\
\OptLine{DPLOT\_STYLE}{TYPE}{VALUES}{DEFAULT}{DESCRIPTION} \\
\OptLine{NUMBER\_DENSITY\_PLOT}{TYPE}{VALUES}{DEFAULT}{DESCRIPTION} \\
}
\Description   {
This command will do a density plot of the Density array read in by the \\
\C{READ\_DPLOT} command.

   If \K{BAR\_LEGEND} is \V{ON} it will plot a gray scale code to the right of 
   the plot.

   \K{BAR\_LEGEND\_PLACES} sets the number of pre- and post-decimal point 
   places for the labelling of the bar legend.

   \K{DPLOT\_GRAYNESS} defines the grayness of the smallest and largest 
   value to be plotted, respectively. If you want small to be 
   white, set the first element larger than the second one; they have to 
   lie between 0 and 1, per PostScript convention. 

   \K{DPLOT\_COLOR} defines the RGB triplets of the smallest and largest 
   RGB value to be plotted, respectively. For example, 0 0 0 1 1 1
   is equivalent to setting \K{DPLOT\_GRAYNESS} to 0 1; 0 0 0 1 0 0 
   is equivalent to spanning from black to red. To span from small
   to large in white to blue, use 1 1 1 0 0 1. Instead of pure blue
   0 0 1, you could use a nicer darker blue of 0.1 0.1 0.7.

   \K{DPLOT\_LINE\_TYPE} defines the line type for the mesh plotted on the 
   density plot.

   \K{DPLOT\_PART} selects the part of the Density array to be plotted. 

   \K{DPLOT\_BOUNDS} sets the real World bounds on the values of the 
   Density array corresponding to the \K{DPLOT\_GRAYNESS} interval.

   \K{DPLOT\_STYLE} selects whether the \K{DPLOT\_BOUNDS} range is to be
   (i) colored in RGB color (\V{COLOR}), (ii) shadowed with various degrees 
   of gray (\V{GRAY}), or (iii) every cell within the range is to be 
   shaded by the first bound of \K{DPLOT\_GRAYNESS} and all other cells by
   the second bound of \K{DPLOT\_GRAYNESS}. 

   If \K{NUMBER\_DENSITY\_PLOT} is set to \V{ON} the number is plotted for each 
   cell instead of the gray rectangle. This number shows the height of 
   the function. You can use the same mechanism as for \V{GRAY} to show only 
   numbers in certain range.

   \K{NUMBER\_PLACES} sets the number of spaces before and after the decimal 
   point when \K{NUMBER\_DENSITY\_PLOT} = \V{ON}.

   \K{POINT\_FONT} sets the font type for the numbers when
   \K{NUMBER\_DENSITY\_PLOT} = \V{ON}.

   \K{PRINT\_FONT} sets the font type for the numbers for the bar legend.

   \K{PRINT\_DXY} will offset the X and Y of the number printed in each cell 
   (in World coordinates).
}




\Command      {CAPTION}{place a caption next to an axis}

\Options{
\OptLine{CAPTION\_POSITION}{TYPE}{VALUES}{DEFAULT}{DESCRIPTION} \\
\OptLine{CAPTION\_FONT}{TYPE}{VALUES}{DEFAULT}{DESCRIPTION} \\
\OptLine{CAPTION\_TEXT}{TYPE}{VALUES}{DEFAULT}{DESCRIPTION} \\
\OptLine{CAPTION\_XLEFT}{TYPE}{VALUES}{DEFAULT}{DESCRIPTION} \\
\OptLine{CAPTION\_XRIGHT}{TYPE}{VALUES}{DEFAULT}{DESCRIPTION} \\
}
\Description  {
This command puts text at pre-specified positions around the plot.

The sequence of \C{CAPTION} commands for the same type of a caption 
is important. The captions
will be placed around the graph starting with the position closest to
the graph. Therefore, a subtitle should be done before the title, but the
X-title should be done before the X-subtitle. \K{CAPTION\_XLEFT} is the
X-position in the Plot coordinates of the leftmost part of the 
Y-captions (by default -0.15). \C{AXES2D} returns the precise
values for \K{CAPTION\_XLEFT}, so use \C{CAPTION} after \C{AXES2D} without
specifying \K{CAPTION\_XLEFT}. \K{CAPTION\_XRIGHT} is a similar variable 
that is used when \K{Y\_SCALE} = \V{RIGHT}.

The following positions with respect to the graph are available:
\begin{description}
\item[1] above graph, centered
\item[2] below graph, centered
\item[3] left of graph, centered, 
\item[4] below graph, right
\item[5] left of graph, top
\item[6] right of graph, center
\item[7] right of graph, top
\item[8] left, top corner of graph
\item[9] right, top corner of graph
\item[10] right, bottom corner of graph
\item[11] left, bottom corner of graph
\item[12] center, top of graph
\end{description}
}





\Command      {RESET\_CAPTIONS}{reset caption positioning}
\Commandline  {RESET\_CAPTIONS}
\Description{Resets the positions for the subsequent captions. 
             This command has to be executed after the \C{WORLD} and 
             before the \C{AXES2D} commands.}




\Command      {LINE2D}{draw a line}

\Options{
\OptLine{LINE2D\_XY1}{TYPE}{VALUES}{DEFAULT}{DESCRIPTION} \\
\OptLine{LINE2D\_XY2}{TYPE}{VALUES}{DEFAULT}{DESCRIPTION} \\
\OptLine{LINE2D\_GRAYNESS}{TYPE}{VALUES}{DEFAULT}{DESCRIPTION} \\
\OptLine{LINE2D\_WIDTH}{TYPE}{VALUES}{DEFAULT}{DESCRIPTION} \\
\OptLine{LINE2D\_WIDTH\_SCALING}{TYPE}{VALUES}{DEFAULT}{DESCRIPTION} \\
\OptLine{LINE2D\_LINE\_TYPE}{TYPE}{VALUES}{DEFAULT}{DESCRIPTION} \\
\OptLine{CLIP}{TYPE}{VALUES}{DEFAULT}{DESCRIPTION} \\
}
\Description  {
Plots a line specified in World coordinates. It uses \K{LINE2D\_LINE\_TYPE} 
if defined (i.e. in 1..LINTYPS, where LINTYPS is the number of line types
read from file `psgl1.ini'), otherwise it uses \K{LINE2D\_WIDTH} and 
\K{LINE\_GRAYNESS}. If \K{LINE2D\_WIDTH\_SCALING} is \V{X}
(case insensitive), the line width is specified in the units of
the X-axis, otherwise in the units of the Y-axis. If \K{CLIP} is
off, the line can occur outside the plot area, otherwise it
is clipped.}



\Command      {SET\_RECORD}{sets RECORD to selected Table elements}

\Options{
\OptLine{ NO\_XY\_SCOLUMNS}{TYPE}{VALUES}{DEFAULT}{DESCRIPTION} \\
\OptLine{XY\_SCOLUMNS}{TYPE}{VALUES}{DEFAULT}{DESCRIPTION} \\
 }
\Description  {
This command puts the selected elements in the first line of
the Table data into the \K{RECORD} variable. \K{RECORD} can then
be used in captioning the plot or plots.

This command is useful in combination with the \C{SELECT\_DATA}
command in order to produce many different captions in one set of 
plots, from one data file.
}




\Command      {PRINT}{print a text}

\Options{
\OptLine{PRINT\_XY}{TYPE}{VALUES}{DEFAULT}{DESCRIPTION} \\
\OptLine{PRINT\_DXY}{TYPE}{VALUES}{DEFAULT}{DESCRIPTION} \\
\OptLine{COORDINATES}{TYPE}{VALUES}{DEFAULT}{DESCRIPTION} \\
\OptLine{PRINT\_FONT}{TYPE}{VALUES}{DEFAULT}{DESCRIPTION} \\
\OptLine{PRINT\_ANGLE}{TYPE}{VALUES}{DEFAULT}{DESCRIPTION} \\
\OptLine{PRINT\_HORIZ}{TYPE}{VALUES}{DEFAULT}{DESCRIPTION} \\
\OptLine{PRINT\_VERT}{TYPE}{VALUES}{DEFAULT}{DESCRIPTION} \\
\OptLine{PRINT\_TEXT}{TYPE}{VALUES}{DEFAULT}{DESCRIPTION} \\
\OptLine{PRINT\_MODE}{TYPE}{VALUES}{DEFAULT}{DESCRIPTION} \\
\OptLine{PRINT\_POINT}{TYPE}{VALUES}{DEFAULT}{DESCRIPTION} \\
\OptLine{XY\_COLUMNS}{TYPE}{VALUES}{DEFAULT}{DESCRIPTION} \\
% \K{PRINT\_MODE}    = \String{1} & \V{POINT} \OR\ \V{XY} \\
}
\Description  {
Prints text positioned in World (\K{COORDINATES} = \V{WORLD}) or 
Plot (\K{COORDINATES} = \V{PLOT}) coordinates
at \K{PRINT\_XY} (\K{PRINT\_MODE} = \V{XY}).
Alternatively, if \K{PRINT\_MODE} = \V{POINT}, the coordinates of the point
\K{PRINT\_POINT} in the columns \K{XY\_COLUMNS} are used. If point index
is out of range, the last point is used. In either case, (X,Y) is
translated for \K{PRINT\_DXY}. 
\K{PRINT\_ANGLE} is a rotation of the text, \K{PRINT\_HORIZ} is 1 for 
left justified, 2 for centered and 3 for right justified, \K{PRINT\_VERT}
is 1 for bottom aligned, 2 for center aligned and 3 for top aligned.

For all text printing, including the one submitted to the \C{CAPTION}
command, the following conventions hold:

\begin{itemize}
\item Superscript: embedded in \_
\item Subscript  : embedded in $\wedge$
\item Greek char : embedded in @
\end{itemize}

Multi-level embedding is allowed.

Special characters can be printed by using the PostScript character codes.
For example, to print the \Ang character, use `\\305'.
}



\Command      {NEW\_PAGE}{start a new page}
\Commandline  {NEW\_PAGE  NO\_COPIES \Integer{1}}
\Description  {Advances to the next page: the next plot will be 
               on the new page. It can only be used when Encapsulated 
               PostScript (EPSF) is not selected.}





\Command      {ARROW}{draw an arrow}
\Commandline  {ARROW [options] \\
   \begin{tabular}{ll}
    ARROW\_POSITION = \Real{4} & X1 Y1 X2 Y2 \\
    ARROW\_SHAPE  = \Real{3} & tail thickness, arrow width, arrow length \\
   \end{tabular}}
\Description  {
Draws an arrow from (X1,Y1) to (X2,Y2) (tail to tip), in World coordinates
(\K{COORDINATES} = \V{WORLD}) or in Plot coordinates (\K{COORDINATES} = 
\V{PLOT}). The elements of \K{ARROW\_SHAPE} are tail thickness, arrow 
width, arrow length in Plot coordinates.}




\Command      {POSTSCRIPT}{write a PostScript command}
\Commandline  {POSTSCRIPT POSTSCRIPT\_TEXT = \String{1}}
\Description {
Takes the \K{POSTSCRIPT\_TEXT} string as a literal PostScript command and
writes it to the output PS file.}





\Command      {TRANSFORM}{transform Table or Density array data}

\Options{
\OptLine{TRF\_TYPE}{TYPE}{VALUES}{DEFAULT}{DESCRIPTION} \\
\OptLine{TRF\_PARAMETERS}{TYPE}{VALUES}{DEFAULT}{DESCRIPTION} \\
\OptLine{TRF\_UNDEFINED}{TYPE}{VALUES}{DEFAULT}{DESCRIPTION} \\
}
\Description  {
Transforms the columns of the Table array selected by \K{XY\_SCOLUMNS}
array. If \K{NO\_XY\_SCOLUMNS} are both 0, it transforms the Density array
instead. You must be careful with labelling the ticks (\C{AXES2D}) and
scaling (\C{WORLD}) when using this option since it transforms the data
itself not only plotting of them.

\K{TRF\_PARAMETERS} sets any parameters that may be required for the 
transformation. 

\K{TRF\_TYPE} selects the type of transformation:

  \begin{description}
    \item[\V{LOGARITHMIC1}:]  $Y = \ln[\mbox{TRF\_PARAM(1)} + 
                            (Y-YMIN) \cdot \mbox{TRF\_PARAM(2)}]$

    \item[\V{LOGARITHMIC2}:]  $Y = \ln[\mbox{TRF\_PARAM(1)} + 
                            Y \cdot \mbox{TRF\_PARAM(2)}]$

    \item[\V{LOGARITHMIC3}:]  $Y = \mbox{log10}[\mbox{TRF\_PARAM(1)} + 
                            (Y-YMIN) \cdot \mbox{TRF\_PARAM(2)}]$

    \item[\V{LOGARITHMIC4}:]  $Y = \mbox{log10}[\mbox{TRF\_PARAM(1)} + 
                            Y \cdot \mbox{TRF\_PARAM(2)}]$

    \item[\V{CUMULATIVE}:]  $Y_i = \sum_{k=1,i} Y_k$

    \item[\V{LINEAR}:]  $Y = \mbox{TRF\_PARAM(1)} + 
                            Y \cdot \mbox{TRF\_PARAM(2)}$

    \item[\V{INVERSE}:]  $Y = \mbox{TRF\_PARAM(1)} + 
                              \mbox{TRF\_PARAM(2) / Y}$

    \item[\V{EXPONENTIAL}:]  $Y = \mbox{TRF\_PARAM(1)} +
                              \exp[\mbox{TRF\_PARAM(2)} + \mbox{TRF\_PARAM(3)}]$

    \item[\V{NORMALIZE}:]  $Y = Y / \sum_i Y_i$
   \end{description}

    YMIN is the smallest value in all selected columns of the 
    Table array or the smallest value in the Density array,
    as appropriate. \V{CUMULATIVE} is not available for 
    transformation of the Density array.

    Any transformation that is undefined (for example, a logarithm 
    of a non-positive argument, or a division by zero) is assigned
    a value \K{TRF\_UNDEFINED}.
}




\Command      {RESET}{reset \TOP}
\Commandline  {RESET}
\Description{
Reads in the {\tt top.ini} file again. Resets all parameters to its default
values. Use \C{RESET} before plotting the second, third, \etc\ plot in the
same \TOP\ program to prevent surprises resulting from the non-default
values of arguments such as \K{X\_TICK}, \K{XY\_COLUMN}, \etc\ which 
were possibly set automatically when producing the first plot.}





\Command      {GET\_BARS}{calculate histogram bars}

\Options{
\OptLine{BAR\_DX}{TYPE}{VALUES}{DEFAULT}{DESCRIPTION} \\
\OptLine{WORLD\_WINDOW}{TYPE}{VALUES}{DEFAULT}{DESCRIPTION} \\
\OptLine{XY\_COLUMNS}{TYPE}{VALUES}{DEFAULT}{DESCRIPTION} \\
% \K{WORLD\_WINDOW}    = \Real{4} & XMIN YMIN XMAX YMAX \\
}
\Description   {
Transforms the current X vector into a histogram by counting how many
X coordinates fall in a certain interval. The interval size must be
specified explicitly by the real World range \K{BAR\_DX}. The centers 
of these intervals
are returned in X and the heights of the bars in Y.  Note that the
current data in the Table array is erased, histogram coordinates
are copied to the current X and Y.
Only XMIN and XMAX are used to determine the X-range. If any of
these two is undefined (-999) default values are supplied as the 
largest and smallest X in the data with an added DX on both sides.
XMAX is corrected so that XMAX = XMIN + (N+1)*BAR\_DX where N is the
number of points (bars) in the new Table array. The first bin starts 
at XMIN+0.5*DX and the last bin ends at XMAX-0.5*DX. For example, 
if you have points with X-coordinates ranging from 0 to 100, you could 
set \K{WORLD\_WINDOW} to -5 0 105 0 and \K{BAR\_DX} to 10, which
will create bins 0--9.999, 10--19.999, ..., 90--100.0; they will
be centered on 5, 15, 25, etc. There will also be half of the bin
width on each side of the histograms when it is plotted (from $-5$ 
to 105 on the X-axis).

\K{XY\_COLUMNS} selects the X and Y columns. If Y column exists before
the call, this routine also calculates the average and standard deviaton
of the Y-values in each bin. These are returned in the two columns
after the currently existing columns.}




\Command      {GET\_DENSITY}{calculate Density plot data}

\Options{
\OptLine{BAR\_DX}{TYPE}{VALUES}{DEFAULT}{DESCRIPTION} \\
\OptLine{BAR\_DY}{TYPE}{VALUES}{DEFAULT}{DESCRIPTION} \\
\OptLine{WORLD\_WINDOW}{TYPE}{VALUES}{DEFAULT}{DESCRIPTION} \\
\OptLine{XY\_COLUMNS}{TYPE}{VALUES}{DEFAULT}{DESCRIPTION} \\
%\K{WORLD\_WINDOW}      = \Real{4} & XMIN YMIN XMAX YMAX \\
}
\Description{
Transforms the current X and Y vectors into a density plot (3D histogram) 
by counting how many X and Y coordinates fall in a certain interval. The 
interval sizes must be specified explicitly by the real parameters 
\K{BAR\_DX} and 
\K{BAR\_DY}. The heights of the density plot are returned in the Density array. 
All four values in \K{WORLD\_WINDOW} are used to determine the X-range. 
If any of them is undefined (-999) default values are supplied as the 
largest and smallest X and Y in the data with an added DX/DY on all sides. 
This makes the WORLD exactly superposable to the world from the 1 .. NBARX,
1..NBARY density plot. This world goes from 0,0 to NBARX+1,NBARY+1. The
center of the first bin is at XMIN+DX,YMIN+DY and the center of the last
bin is at XMAX-DX,YMAX-DY. This gives a 0.5*DX, 0.5*DY margin around
the plotted bins.
XMAX (and YMAX) is corrected so that XMAX = XMIN + (N+1)*BAR\_DX where 
N is the number of points (bars) in the new Density array.
\K{XY\_COLUMNS} selects the X and Y columns.
}



\Command      {SHUFFLE\_DPLOT}{re-organize the Density plot data}

\Options{
\OptLine{SHUFFLE\_OPERATION}{TYPE}{VALUES}{DEFAULT}{DESCRIPTION} \\
% \K{SHUFFLE\_OPERATION}  = \String{1} & TRANSPOSE \OR\ REVERT\_X \OR\ 
%                                      REVERT\_Y \\
}
\Description{
The transpose operation only has sense for square matrices. The other
two commands invert the order of the points along the X and Y coordinates, 
respectively. To exchange the orientation of the X and Y axes as they
appear on the plot, you should use the \K{DPLOT\_ORIENTATION} option
of the \C{READ\_DPLOT} command.}




\Command      {DENSITY\_TO\_XY}{Density to XY data}
\Commandline  {DENSITY\_TO\_XY \\
}
\Description{
Copies the current density array to the XY table.
}



\Command      {XY\_TO\_DENSITY}{XY to density data}
\Commandline  {XY\_TO\_DENSITY \\
}
\Description{
Copies the current XY table to the density array.
}



\Command      {LEGEND}{draw a legend}

\Options{
\OptLine{SYMBOL}{TYPE}{VALUES}{DEFAULT}{DESCRIPTION} \\
\OptLine{LEGEND\_POSITION}{TYPE}{VALUES}{DEFAULT}{DESCRIPTION} \\
\OptLine{PLOT2D\_LINE\_TYPE}{TYPE}{VALUES}{DEFAULT}{DESCRIPTION} \\
\OptLine{PLOT2D\_SYMBOL\_TYPE}{TYPE}{VALUES}{DEFAULT}{DESCRIPTION} \\
\OptLine{BAR\_LINE\_TYPE}{TYPE}{VALUES}{DEFAULT}{DESCRIPTION} \\
\OptLine{BAR\_GRAYNESS}{TYPE}{VALUES}{DEFAULT}{DESCRIPTION} \\
\OptLine{DESCRIPTION}{TYPE}{VALUES}{DEFAULT}{DESCRIPTION} \\
\OptLine{DESCRIPTION\_FONT}{TYPE}{VALUES}{DEFAULT}{DESCRIPTION} \\
% \K{SYMBOL}   = \String{1} & \V{LINE}  \OR\  \V{POINT}  \OR\  \V{BAR}    \\
}
\Description{
Plots the legend for objects (curves, points, and bars).
The legend is in the form 'symbol symbol\_description'.
Successive \C{LEGEND} commands print legends one per line, starting at the
top. \C{RESET\_LEGEND} has to be issued before \C{LEGEND}. Note that you can
use this command to place text right to the plot if you specify
invisible object types (0). The value of the \K{SYMBOL} 
keyword determines the type of the object symbol in the legend (POINT, LINE,
or BAR).
\K{DESCRIPTION} is the text describing the symbol. \K{LEGEND\_POSITION} can
have the same values as \K{CAPTION\_POSITION}.}




\Command      {RESET\_LEGEND}{reset legend positioning}
\Commandline  {RESET\_LEGEND}
\Description  {Resets the starting Y position of the next legend
               to the initial value at the top of the plot.}





\Command      {PLOT\_ERROR\_BARS}{draw error bars}

\Options{
\OptLine{XY\_COLUMNS}{TYPE}{VALUES}{DEFAULT}{DESCRIPTION} \\
\OptLine{ERROR\_COLUMN}{TYPE}{VALUES}{DEFAULT}{DESCRIPTION} \\
\OptLine{ERROR\_SIGN}{TYPE}{VALUES}{DEFAULT}{DESCRIPTION} \\
\OptLine{PLOT2D\_SYMBOL\_TYPE}{TYPE}{VALUES}{DEFAULT}{DESCRIPTION} \\
\OptLine{PLOT2D\_LINE\_TYPE}{TYPE}{VALUES}{DEFAULT}{DESCRIPTION} \\
\OptLine{POINT\_FONT}{TYPE}{VALUES}{DEFAULT}{DESCRIPTION} \\
}
\Description{
This command plots error bars. It relies on X and Y columns selected
by the \K{XY\_COLUMNS} argument (as in the PLOT2D command), and on the
\K{ERROR\_COLUMN} argument which selects the column in the Table array that
contains the error in Y-coordinate for each (X,Y) point. The graphical
attributes are the same as those used for the \C{PLOT2D} command.  The
error-bar is a vertical line with two short horizontal caps on top and
bottom. The error line extends for `error' units below the central
point and for `error' units above the central point; thus, the total
height of the error bar is twice the error. \K{PLOT2D\_SYMBOL\_TYPE} 
selects the symbol type used to plot at the center of the error bar.
If \\ \K{PLOT2D\_SYMBOL\_TYPE} is $-1$ then \K{POINT\_FONT} font
is used to print the points indices instead of point symbols.
\K{PLOT2D\_LINE\_TYPE} selects the line type used for plotting the
error bars. IF \K{ERROR\_SIGN} contains word \V{UP}, the error value will
be added to the y-value. If it contains \V{DOWN}, the error value
will be subtracted from the y-value. \K{ERROR\_SIGN} contains both 
by default, \V{UP DOWN}.}




\Command      {SELECT\_DATA}{select some rows in Table array}
\Commandline  {SELECT\_DATA SELECT\_COLUMN = \Integer{1},
                            SELECT\_RANGE = \Real{2}}
\Description  {
This command selects the lines of the Table array that have a value
in a specified column in a specified range. This is useful when you
want to plot only a subset of data, for example make a (X,Y) plot at
a fixed Z.}




\Command      {SWITCH\_PS}{open a new PostScript file}
\Commandline  {SWITCH\_PS FILE = \String{1}}
\Description  {This command will close the current PS file and open
               another one for subsequent output.}




\Command      {FIT}{non-linear least-squares fit of data}

\Options{
\OptLine{XY\_COLUMNS}{TYPE}{VALUES}{DEFAULT}{DESCRIPTION} \\
\OptLine{ERROR\_COLUMN}{TYPE}{VALUES}{DEFAULT}{DESCRIPTION} \\
\OptLine{FIT\_MODEL}{TYPE}{VALUES}{DEFAULT}{DESCRIPTION} \\
\OptLine{FIT\_PARAM\_INITIAL}{TYPE}{VALUES}{DEFAULT}{DESCRIPTION} \\
\OptLine{FIT\_PARAM\_INDICES}{TYPE}{VALUES}{DEFAULT}{DESCRIPTION} \\
\OptLine{FIT\_CUTOFFS}{TYPE}{VALUES}{DEFAULT}{DESCRIPTION} \\
\OptLine{FIT\_WORLD}{TYPE}{VALUES}{DEFAULT}{DESCRIPTION} \\
\OptLine{WORLD\_WINDOW}{TYPE}{VALUES}{DEFAULT}{DESCRIPTION} \\
\OptLine{NO\_SPLINE\_POINTS}{TYPE}{VALUES}{DEFAULT}{DESCRIPTION} \\
% \K{FIT\_MODEL}  = \String{1} & \V{POLYNOMIAL} \OR\ \V{NORMAL} \OR\ \\
%                           & \V{LOG-NORMAL} \OR\ \V{EXPONENTIAL} \OR\ \\
%                           & \V{EXPONENTIAL2} \OR\ \V{EXPONENTIAL3} \OR\ \\
%                           & \V{LOGARITHMIC} \OR\ \\
%                           & \V{POLYGAUSS} \OR\ \V{POLYGAUSS360} \OR\ \\
%                           & \V{POWER} \OR\ \V{DISSOCIATION} \\
}
\Description{This command does a non-linear least-squares fitting of the
(X,Y) data in columns \K{XY\_COLUMNS} to the model selected by \K{FIT\_MODEL}.
The standard errors of the data points can be specified in column 
\K{ERROR\_COLUMN}.
If this column index is undefined, all errors are set to 1. 
\K{FIT\_PARAM\_INITIAL} specifies the initial estimates for all the parameters.
\K{FIT\_PARAM\_INDICES} specifies the indices of parameters to be
actually varied in the fitting. \K{FIT\_CUTOFFS} specifies the convergence
criterion: when the absolute and relative change in the chi-square is
smaller than specified in two consecutive cycles, the fitting exits.
If \K{FIT\_WORLD} is \V{OFF}, the Y-column contains the function values for 
the corresponding X.
If \K{FIT\_WORLD} is \V{ON}, the X column is changed to \K{NO\_SPLINE\_POINTS}
points that equidistantly span \K{WORLD\_WINDOW} X-range and Y-column
is filled with the function values calculated at these new X-points. Note
that Xmin and Xmax of the World are specified as the first and third
element of \K{WORLD\_WINDOW}.
}



\Command      {FIT2}{non-linear least-squares fit of data}
\Commandline  {FIT2 [options] \\
}
\Description{Downward compatible with \C{FIT}, except that it can also do
multidimensional least-squares fitting. Should replace \C{FIT} with time.
}




\Command      {SMOOTH\_TABLE}{smooth the Table array data}

\Options{
\OptLine{SMOOTH\_TYPE}{TYPE}{VALUES}{DEFAULT}{DESCRIPTION} \\
\OptLine{NO\_SPLINE\_POINTS}{TYPE}{VALUES}{DEFAULT}{DESCRIPTION} \\
\OptLine{XY\_COLUMNS}{TYPE}{VALUES}{DEFAULT}{DESCRIPTION} \\
\OptLine{NO\_XY\_SCOLUMNS}{TYPE}{VALUES}{DEFAULT}{DESCRIPTION} \\
\OptLine{XY\_SCOLUMNS}{TYPE}{VALUES}{DEFAULT}{DESCRIPTION} \\
\OptLine{HALF\_WINDOW}{TYPE}{VALUES}{DEFAULT}{DESCRIPTION} \\
\OptLine{PLOT\_POINTS}{TYPE}{VALUES}{DEFAULT}{DESCRIPTION} \\
% \K{SMOOTH\_TYPE}  = \String{1}  & \V{SPLINE} \OR \V{AVERAGE} \\
}
\Description{Smoothes data in the Table array using one of the two methods
selected by \K{SMOOTH\_TYPE}: the cubic spline or running average method.
The main appearance difference between the two methods is that spline
smoothing goes through the original points (if they were evenly
distributed), whereas window averaging changes the original points.

In the spline method, \K{NO\_SPLINE\_POINTS} is the number of points to
be calculated by spline smoothing. It is returned in \K{PLOT\_POINTS}.
Columns X and Y are selected by \K{XY\_COLUMNS}; note the difference with
the running average method.

In the running average method, the weight of point $j$ is proportional
to $(\K{HALF\_WINDOW}-|i-j|+1)$ where $i$ is
the central point and \K{HALF\_WINDOW} is the number of points left and right
of point $i$ that are in the window. Only the columns specified by
\K{NO\_XY\_SCOLUMNS} and \K{XY\_SCOLUMNS} are smoothed; note the difference
with the spline method. \K{PLOT\_POINTS} is not changed.
}




\Command      {READ\_PDB}{read a Brookhaven molecular structure}
\Commandline  {READ\_PDB FILE = \String{1}, PDB\_FORMAT = \String{1}}
\Description  {Reads in the PDB file according to the format specified.
               Format can be \V{PDB} or \V{XYZ}. The former is the 
               Brookhaven Protein Databank format. The latter consists
               only of x,y,z for one atom per line.}




\Command      {WRITE\_PDB}{write a Brookhaven molecular structure}
\Commandline  {WRITE\_PDB FILE = \String{1}}
\Description  {Writes out the PDB file.}




\Command      {MAKE\_BONDS}{make a list of bonds}

\Options{
\OptLine{BOND\_TYPE}{TYPE}{VALUES}{DEFAULT}{DESCRIPTION} \\
\OptLine{COVALENT\_BOND}{TYPE}{VALUES}{DEFAULT}{DESCRIPTION} \\
\OptLine{BOND\_COLOR}{TYPE}{VALUES}{DEFAULT}{DESCRIPTION} \\
\OptLine{BOND\_LINE}{TYPE}{VALUES}{DEFAULT}{DESCRIPTION} \\
\OptLine{BOND\_TAPER}{TYPE}{VALUES}{DEFAULT}{DESCRIPTION} \\
\OptLine{BOND\_WIDTH\_FACTO}{TYPE}{VALUES}{DEFAULT}{DESCRIPTION} \\
\OptLine{ADD\_BONDS}{TYPE}{VALUES}{DEFAULT}{DESCRIPTION} \\
% \K{BOND\_TYPE}  = \String{1} &  \V{COVALENT} \OR\ \V{SEQUENTIAL} \\
}
\Description{Constructs the bonds between the currently selected bonding
set of atoms and either adds them to the list of existing bonds or creates
this list from scratch, depending on the logical variable \K{ADD\_BONDS}.

\K{BOND\_COLOR} sets the color of the current bonds.
The coloring scheme is from PostScript: 0 for
black and 1 for white; intermediate values select various shades of gray.

\K{COVALENT\_BOND} defines the distance cuttoff between which the atom
pair is recognized as bonded to each other. If any of the two values is
undefined (i.e. -999), then the covalent bond exists when a distance
between the two atoms is less than 0.55 times the sum of the two van der 
Waals radii and more than half of that value. The radii are obtained
from the PDB atom names by first recognizing an extended atom type in
the residue, and then looking into the radii library file for all
atom types in all residue types. Hydrogen atoms are treated separetely 
--- they are just recognized as such directly and their radii
assigned accordingly.

\K{BOND\_LINE} sets the line type for drawing the current bonds.
If the line type is negative than only the line of the type
-\K{BOND\_LINE} is drawn, not the polygon with caps.

\K{BOND\_WIDTH\_FACTOR} sets the bond width. If too large (approx 1),
bond coordinates may be undefined.

\K{BOND\_TAPER} is a factor that is multiplied by the delta Z of the
bond and its width to increase the width of the nearer end, and decrease
the width of the further end. It should be 0 or larger than 0.
}




\Command      {LABEL\_ATOMS}{label the selected atoms}

\Options{
\OptLine{LABEL\_STYLE}{TYPE}{VALUES}{DEFAULT}{DESCRIPTION} \\
\OptLine{LABEL\_FONT}{TYPE}{VALUES}{DEFAULT}{DESCRIPTION} \\
\OptLine{LABEL\_LOCATION}{TYPE}{VALUES}{DEFAULT}{DESCRIPTION} \\
% \K{LABEL\_STYLE} = \String{1} & [\V{ATOM\_TYPE} \OR\ \V{RESIDUE\_TYPE1} \OR\ \\
%                             &  \V{RESIDUE\_TYPE3} \OR\ \V{ATOM\_INDEX} \OR\ \\
%                             &  \V{RESIDUE\_NUMBER} \OR\ \V{RESIDUE\_INDEX}] \\
% \K{LABEL\_LOCATION} = \Integer{1} & 1 ... label centered on the atom \\
%                                 & 2 ... label right to the atom \\
}
\Description  {Labels the currently selected labelling set of atoms with the
currently selected labelling style. Labelling style is a string and
can be any sequence of the items listed above, in any order.}




\Command      {DEFAULT\_ATOM\_COLOR}{color all the atoms}
\Commandline  {DEFAULT\_ATOM\_COLOR}
\Description  {Assigns default attributes to all the atoms: atom color
and atom lines. Useful to do after \C{SELECT\_ATOMS} which assigns
a specific color and line; if you need generic coloring for different
atom types.
}





\Command      {SELECT\_ATOMS}{select atoms in a molecule}

\Options{
\OptLine{SELECTION\_SEARCH}{TYPE}{VALUES}{DEFAULT}{DESCRIPTION} \\
\OptLine{FROM}{TYPE}{VALUES}{DEFAULT}{DESCRIPTION} \\
\OptLine{FOR}{TYPE}{VALUES}{DEFAULT}{DESCRIPTION} \\
\OptLine{RES\_TYPE}{TYPE}{VALUES}{DEFAULT}{DESCRIPTION} \\
\OptLine{ATOM\_TYPE}{TYPE}{VALUES}{DEFAULT}{DESCRIPTION} \\
\OptLine{SELECTION\_MODE}{TYPE}{VALUES}{DEFAULT}{DESCRIPTION} \\
\OptLine{SELECTION\_STATUS}{TYPE}{VALUES}{DEFAULT}{DESCRIPTION} \\
\OptLine{RADIUS}{TYPE}{VALUES}{DEFAULT}{DESCRIPTION} \\
\OptLine{RADIUS\_FACTOR}{TYPE}{VALUES}{DEFAULT}{DESCRIPTION} \\
\OptLine{ATOM\_COLOR}{TYPE}{VALUES}{DEFAULT}{DESCRIPTION} \\
\OptLine{ATOM\_LINE}{TYPE}{VALUES}{DEFAULT}{DESCRIPTION} \\
\OptLine{SELECTION\_SEGMENT}{TYPE}{VALUES}{DEFAULT}{DESCRIPTION} \\
\OptLine{SELECTION\_STEP}{TYPE}{VALUES}{DEFAULT}{DESCRIPTION} \\
\OptLine{SPHERE\_CENTER}{TYPE}{VALUES}{DEFAULT}{DESCRIPTION} \\
\OptLine{SLAB}{TYPE}{VALUES}{DEFAULT}{DESCRIPTION} \\
% \K{SELECTION\_SEARCH} = \String{1} & \V{SPHERE}  \OR\ \V{SEGMENT} \\
% \K{FROM}              = \String{1} & \V{ALL} \OR\ \V{DISPLAYING} \OR\ \V{BONDING} \OR\
%                         \V{LABELLING} \\
% \K{FOR}               = \String{1}    & \V{DISPLAYING} \OR\ \V{BONDING} \OR\
%                          \V{LABELLING} \\
% \K{RES\_TYPE}         = \String{0} & \V{ALL}  \OR\ three letter codes \\
% \K{ATOM\_TYPE}        = \String{0} & \V{ALL} \OR\ \V{SDCH} \OR\ \V{MNCH}
%                         \OR\ atom names \\
% \K{SELECTION\_MODE}   = \String{1} & \V{ATOM} \OR\ \V{RESIDUE} \\
% \K{SELECTION\_STATUS} = \String{1}   & \V{ADD} \OR\ \V{REMOVE} \OR\
%                         \V{INITIALIZE} \\
%\multicolumn{2}{l}{SEARCH = \V{SEGMENT}:} & \\
% \K{SELECTION\_SEGMENT} = \String{2} & \#RES1 \#RES2 \\
%\multicolumn{2}{l}{SEARCH = \V{SPHERE}:} & \\
% \K{SPHERE\_CENTER}     = \String{2} & \#RES1 ATOM\_NAME \\
% \K{SLAB}       = \Real{5}   & dz1 dz2 xtrans ytrans ztrans    \\
}
\Description{Adds atoms to, removes atoms from, or initializes the \ASGL\
atom sets selected by \K{FOR}. \K{SELECTION\_STATUS} determines whether
the selected atoms are added (\V{ADD}), removed (\V{REMOVED}), or a set is
initialized and then the selected atoms are added (\V{INITIALIZE}). There
are three sets of atoms in \ASGL: (1) the atoms that are selected for
display (\C{BALL\_STICK}), (2) the atoms that are selected for calculation
of bonds (\C{MAKE\_BONDS}) and (3) the atoms that are selected for labelling
(\C{LABEL\_ATOMS}). The \K{FOR} variable is a scalar string that can contain
any combination of the three selections: The selection of atoms is a
hierarchical process: First, \K{FROM} specifies which of the three atom
sets are used for scanning; if \K{FROM} = \V{ALL}, all atoms in the input
pdb file are used for scanning. \K{ATOM\_TYPE} and \K{RES\_TYPE} constrain
scanning to specified atom and residue types, respectively. They can
contain several atom and residue types in one quoted string. \K{SELECTION\_MODE}
determines whether only an atom satisfying all search criteria is to
be selected or all atoms in a residue of a found atom are to be selected.

The \K{SEARCH\_STRING} specifies \V{SEGMENT} or \V{SPHERE} search; this
determines the other possible arguments.

The \V{SEGMENT} search scanns only a single segment specified by the
beginning and ending residue number (as found in the input atom file),
\K{SEGMENT\_RANGE}. The value of the residue number can be \V{X}, which
implies the first or last residue, as appropriate.  \K{SELECTION\_STEP}
is a step in the residue index used in scanning for the atoms. This is
useful in labelling only every 5-th CA atom, for example.

The \V{SPHERE} search scanns only over those atoms that are closer than
\K{SPHERE\_RADIUS} to the \K{SPHERE\_CENTER} atom, after the center
atom was translated by (xtrans, ytrans, ztrans) specified in \K{SLAB}.
If the first element of \K{SPHERE\_CENTER} is \V{INDEX} then the second element
is an atom index of the center atom; otherwise, the first and second
element are the residue number (as in the input atom file) and the atom
type, respectively. \K{SLAB} specifies the interval on Z-axis relative
to Z of the translated central atom that imposes another condition on
the selected atoms: $Zcen+dz1 < Z+ztrans < Zcen+dz2$. This is useful
to make less crowded plots. Larger Z is on front, so dz1 specifes the
plane that is further away than the dz2 plane. To get any atoms,
$dz1 < dz2$.

The radii of the currently selected display set of
atoms is set to \K{RADIUS} * \K{RADIUS\_FACTOR}. If the final
atom radii is 0, the atom is not drawn. If \K{RADIUS} is undefined
(-999) then the van der Waals radius of that atom type is used.

\K{ATOM\_COLOR} sets the color of the currently selected display set of atoms.
The coloring scheme is from PostScript: 0 for
black and 1 for white; intermediate values select various shades of gray.

\K{ATOM\_LINE} sets the line type for drawing the currently selected
display set of atoms.
}



\Command      {BALL\_STICK}{draw a molecule}

\Options{
\OptLine{PERSPECTIVE}{TYPE}{VALUES}{DEFAULT}{DESCRIPTION} \\
\OptLine{EYE\_TO\_SCREEN}{TYPE}{VALUES}{DEFAULT}{DESCRIPTION} \\
\OptLine{SCREEN\_TO\_TOP}{TYPE}{VALUES}{DEFAULT}{DESCRIPTION} \\
}
\Description  {Plots a ball-and-stick plot of the currently selected
               display set of atoms and of the selected bonds.}




\Command      {ROTATE\_MOL}{rotate the molecule using rotation matrix}
\Commandline  {ROTATE\_MOL ROTATION\_MATRIX = \Real{9}}
\Description  {Rotates the whole molecule according to the rotation matrix.}





\Command      {ROTATE\_MOL\_AXIS}{rotate the molecule around the axis}
\Commandline  {ROTATE\_MOL\_AXIS AXIS \Real{3}, DEGREES = \Real{1}}
\Description  {Rotates the whole molecule around the axis \K{AXIS} for
               \K{DEGREES} degrees.}



\Command      {TRANSLATE\_MOL}{translate the molecule}
\Commandline  {TRANSLATE\_MOL SHIFT = \Real{3}}
\Description  {Translates the whole molecule for \K{SHIFT} \AA ngstroms.}


\Command      {CENTER\_MOL}{center the molecule}
\Commandline  {CENTER\_MOL}
\Description  {Center the whole molecule.}
