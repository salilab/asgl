\section{What is \ASGL ?}


\begin{description}

\item[Purpose:] 
      Easy creation of PostScript files containing scientific plots.


\item[Capabilities:] \ 

\begin{itemize}
\item 2D plots: 
      scatter plots, curve plots, smoothed curve plots, transformation 
      of axes, histograms, spectra, general axes, and plot labelling;

\item 3D plots: density plots;

\item molecule plots: 
      ball-and-stick plots of a molecule in a Brookhaven Protein Databank
      format file; general labelling; virtual Calpha-Calpha 
      bonds; stereo plots;

\item the attributes of the objects to be drawn can be varied by the user;
   for example, the grayness and width of lines, the size of fonts,
   the grayness of histogram bars, \etc.
\end{itemize}

    
\item[Use:] \ 

\ASGL\ is used via a steering file. A steering file contains commands 
specialized for creation of plots. This file is interpreted by \TOP\ which 
then calls appropriate Fortran subroutines to create the PostScript output
file. See Chapter~\ref{CHAPTERTOP} for description of \TOP.
\end{description}


\newpage


\section{Using \ASGL\ with the \TOP\ steering file}

The \ASGL\ program is run by:

{\tt asgl job}

\noindent where {\tt job} is the root of the steering file --- the actual 
steering file must have an extension {\tt .top}. By default, the output of 
the program is then written to the {\tt job.ps} file that can be printed on the
PostScript printer, previewed with a PostScript previewer, modified,
or included in other documents, such as \TeX\ files.

\ASGL\ reads PostScript definitions of line types, font types and
symbols from text file {\tt src/psgl1.ini}. This file can be edited to
extend the capabilities of \ASGL. For example, line widths, grayness,
dash-dot combinations, font types and sizes, symbol types and sizes
can be customized in this way.

There are four coordinate systems (or windows) used by \ASGL. The Base
PostScript system corresponds to the PostScript device and has the
Bounding Box (0,0,612,792). The Paper coordinate system is used 
to specify the Bounding Box and its orientation of the window 
on the paper that will contain the plot. It overlaps with
the Base PostScript system except that it uses cm; its Bounding Box
is (0,0,21.59,27.94). The World coordinate system is defined by 
the data to be plotted. It overlaps with the window specified in 
the Paper coordinate system. The Plot coordinate system is used for 
creating the plot. The origin of the Plot coordinate system is always (0,0),
XMAX is always 1, and YMAX is such that it retains the X/Y aspect 
of the Paper coordinate system. It overlaps with the window specified
in the Paper and World coordinate systems.

Probably the easiest way to use \ASGL\ is to take an example \TOP\ file
from {\tt examples} sub-directory that is closest to what you need and
edit it to suit your requirements exactly. The example \TOP\ files
and their output are given in Chapter~\ref{CHAPTEREXAMPLES}.

A complete list of \ASGL\ commands is given in Chapter~\ref{CHAPTERASGL}.
